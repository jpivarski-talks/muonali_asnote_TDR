The Millepede-based algorithm implemented for global alignment is a
particular case of the most general Millepede algorithm already
referred to in section~\ref{sec:localdt}.  In this case, the ability
to fit tracks and alignment parameters together is parameterized, such
that the algorithm can either let them both float freely (full
Millepede), or align chambers to a fixed set of tracks (alignment to
reference).

To align chambers in CRAFT, the algorithm was tuned to use a fixed set
of tracks from the tracker.  These tracks were propagated into the
muon system, and unbiased residuals distributions were minimized by
aligning chambers.  In the language of section~\ref{sec:localdt}, the
matrices $B_j \to 0$ and $\delta \vec{p}_j$ are no longer parameters
in the fit.  This also decorrelates alignment parameters in different
chambers, greatly simplifying the alignment.

As described
in section~\ref{sec:hipalgo}, the scattering processes that take place in the iron yoke between chambers
deflect muons in a correlated pattern for all the hits in a same chamber. To account
for this correlation, single layer hits and the track intersections associated to them
are combined in a linear fit producing four observables for each chamber: the position
residuals $\Delta x$, $\Delta y$ and the direction
residuals $\Delta \frac{dx}{dz}$ and $\Delta \frac{dy}{dz}$.
Just as in the HIP case, these residuals are related to the 6 alignment parameters $\delta_x$,
$\delta_y$, $\delta_z$, $\phi_x$, $\phi_y$ and $\phi_z$, by Equation~\ref{eqn:dtmatrix}.

The form of the $\chi^2$ associated with the general Millepede algorithm was shown in Equation~\ref{eqn:chi2millepede}.
In this implementation of the algorithm, $\chi^2$ has the form     
\begin{equation}
{\chi^2}^{\mbox{\scriptsize \, track-based}} = \sum_j \left(\Delta \vec{x} - A \cdot \vec{\delta} \right)^T \, \left({\sigma_{\mbox{\scriptsize resid}_{\mbox{\scriptsize $i$}}}}^2\right)^{-1} \, \left(\Delta \vec{x} - A \cdot \vec{\delta} \right)
\label{eqn:globalmpchi2}
\end{equation}
where the $\Delta \vec{x}$ is a 4-component vector, and
$A$ is the matrix which relates the residuals with the alignment parameters. The 
$\left({\sigma_{\mbox{\scriptsize resid}_{\mbox{\scriptsize $i$}}}}^2\right)^{-1}$ matrix
is the inverse of the covariance calculated from the linear fit estimation of the segments. It
includes correlations between positions and directions in both projections.

To avoid the influence of the tails of the residuals distributions, a
filtering operation is performed.  This is especially important
because the tails of the distributions may not be perfectly symmetric.
To set cut boundaries for the residuals on each chamber, the peak is
fitted to a Lorentz function and residuals are required to be within
the full width at half maximum (between $x_0 - \Gamma/2$ and $x_0
+ \Gamma/2$ where $x_0$ and $\Gamma$ are the peak and full-width at
half maximum, respectively).  In a one-dimensional application, this
alignment fit would correspond to a truncated mean.
