In this note, we have demonstrated a variety of software procedures for
aligning different parts of the muon system: layers in DT chambers,
CSC chambers in rings, and global positions of DT chambers relative to
the central tracker, using charged tracks.  We have fully exploited the available data,
horizontal LHC beam-halo muons and vertical cosmic rays.

These procedures will be used without major modifications to re-align
the muon system with muons from LHC collisions, once such data are
available.  The azimuthally-symmetric and broad pseudorapidity
distribution of collisions muons will allow us to extend the global
alignment to all muon chambers.  Although the endcap chambers are
sensitive to only three parameters, $\delta_{r\phi}$,
$\delta_{\phi_y}$ and $\delta_{\phi_z}$ (the same as local CSC
alignment), these parameters can be determined with 400~$\mu$m,
0.4~mrad, and 0.6~mrad resolution with similar results for all degrees
of freedom in the muon barrel, according to 50~pb$^{-1}$ simulations.
To constrain the remaining CSC parameters, we can exploit the
complementarity of the endcap hardware alignment
system~\cite{ref:hardware_alignment} (which measure $\delta_z$ and
$\delta_{\phi_x}$ directly), and independently test its validity in
$\delta_{r\phi}$.

Moreover, the addition of collisions muons and larger beam-halo
datasets will allow new cross-checks to be performed, as the local CSC
alignment (section~\ref{sec:localcsc}) and CSC layer alignment can be
performed with both collisions and beam-halo, so we can doubly
cross-check by performing the same method with different track sources
and different methods with the same track source.  Similarly,
collisions and cosmic rays can both be used in the barrel, and they
differ in how the reference tracks sample the tracker, and hence
potental input track bias.

By verifying the muon alignment in as many ways as possible, we can
add confidence to the muon momentum resolution at all energy scales,
improving sensitivity to signatures of new physics.
